%=====================================================================
\chapter{The Gyrokinetic Model}\label{chap.gkmodel}

%---------------------------------------------------------------------
\section{Foundations and Notation}\label{sec.gkintro}

The definitive works in terms of the complete derivation of the 
gyrokinetic-Maxwell equations for evolution of fluctuations, and 
the neoclassical equations for evaluation of collisional transport,
are due to Sugama and coworkers \cite{sugama:1997,sugama:1998}.
For historical reasons, the notation and presentation here differ 
in some respects from these papers.  Nevertheless, what is 
documented here should be completely consistent with the 
formulae given in Ref.~\cite{sugama:1998}.  We prefer when 
applicable to use Gaussian CGS units\footnote{
See {\tt http://wwwppd.nrl.navy.mil/nrlformulary/NRL\_FORMULARY\_06.pdf}}.  
Roughly speaking, the gyrokinetic approach \cite{antonsen:1980,catto:1981,frieman:1982} 
is based on the assumption that equilibrium quantities are slowly varying,
while perturbations are smaller but more rapidly varying in space.  To 
describe this ordering, it is convenient to introduce the parameters
%
\begin{align}
c_s = &~\sqrt{\frac{T_e}{m_i}} \qquad \text{ion-sound speed} \\
\rho_s = &~\frac{c_s}{\Omega_{ci}} 
     \qquad \text{ion-sound gyroradius} 
\end{align}

\noindent
where $\Omega_{ci}$ is the ion cyclotron frequency. 
The derivation of the gyrokinetic equation proceeds by 
expanding the primitive Fokker-Planck equation in 
the small parameter, $\rho_* \doteq \rho_{\rm s}/a$, 
where $a$ is the plasma minor radius.  

\section{Reduction of the Fokker-Planck Equation}
 
The details of the derivation of the gyrokinetic (and neoclassical) 
equations is beyond the scope of this manual.  Still, we attempt to 
sketch the essential details. The Fokker-Planck equation provides the 
fundamental theory for plasma equilibrium, fluctuations, transport.
In this section, we use Sugama's notation.  The FP equations is written as
%
\begin{align}
\left[ \frac{\partial}{\partial t} + {\bf v}\cdot\nabla + 
\frac{e_a}{m_a} \left( (\E+\Eh) + \frac{\bf v}{c} \times 
(\B+\Bh) \right) \cdot \frac{\partial}{\partial{\bf v}} \right] 
(f_a+{\hat f}_a)
= C_a(f_a+{\hat f}_a) + S_a
\end{align}
%
where $f_a$ is the ensemble-averaged distribution, ${\hat f}_a$ is the 
fluctuating distribution, $S_a$ are sources (beams, RF, etc), and
%
\begin{equation}
C_a = \sum_b C_{ab}(f_a+{\hat f}_a,f_b+{\hat f}_b) 
\end{equation}
%
is the nonlinear collision operator.  The general approach is to separate 
the FP equation into ensemble-averaged, ${\cal A}$, and fluctuating, 
${\cal F}$, components:
%
\begin{align}
\fpa = &~\left. \frac{d}{dt} \right|_{\rm ens} f_a 
 - \langle C_a \rangle_{\rm ens} - D_a - S_a \; , \\ 
\fpf = &~\left. \frac{d}{dt} \right|_{\rm ens} {\hat f}_a 
 +\frac{e_a}{m_a} \left( \Eh + \frac{\bf v}{c} \times \Bh \right) 
 \cdot \frac{\partial}{\partial{\bf v}} ( f_a + {\hat f}_a )
 - C_a + \langle C_a \rangle_{\rm ens} + D_a \; ,
\end{align}
%
where 
%
\begin{equation}
\left. \frac{d}{dt} \right|_{\rm ens} \doteq 
 \frac{\partial}{\partial t} + {\bf v}\cdot\nabla 
+ \frac{e_a}{m_a} \left( \E + \frac{\bf v}{c} \times \B \right) 
 \cdot \frac{\partial}{\partial{\bf v}} \; ,
\end{equation}
%
\begin{equation}
D_a \doteq - \frac{e_a}{m_a} \left<
 \left( \Eh + \frac{\bf v}{c} \times \Bh \right) \cdot 
 \frac{\partial{\hat f}_a }{\partial{\bf v}} \right>_{\rm ens} \; .
\end{equation}
%
such that $D_a$ is the {\it fluctuation-particle interaction operator}.
Ensemble averages are expanded in powers of $\rho_*$ as
%
\begin{align}
f_a = &~f_{a0} + f_{a1} + f_{a2} + \ldots \; , \\ 
S_a = &~\qquad\qquad\qquad S_{a2} + \ldots \; \mbox{(transport ordering)}, \\
\E = &~\E_0 + \E_1 + \E_2 + \ldots \; , \\ 
\B = &~\B_0 \; .
\end{align}
%
Fluctuations are also expanded in powers of $\rho_*$ as
%
\begin{align}
{\hat f}_a = &~{\hat f}_{a1} + {\hat f}_{a2} + \ldots \; , \\
\Eh = &~\Eh_1 +  \Eh_2 + \ldots \; , \\
\Bh = &~\Bh_1 + \Bh_2 + \ldots \; .
\end{align}
%
\subsection{Lowest-order constraints}

The lowest-order ensemble-averaged equation gives the constraints 
%
\begin{equation}
\fpa_{-1} = 0: \qquad \E_0 + \frac{1}{c} {\bf V}_0 \times \B = 0 
\qquad\mbox{and}\qquad
\frac{\partial f_{a0}}{\partial \xi} = 0 
\end{equation}
%
where $\xi$ is the gyroangle.  The only zeroth-order flow 
(i.e., sonic) flow that persists on the fluctuation timescale 
is a purely toroidal flow \cite{hinton:1985}, which we write as
%
\begin{equation}
{\bf V}_0 = V_0 {\bf e}_\varphi = R \, \omega_0(\psi) {\bf e}_\varphi \quad\mbox{where}
\quad \omega_0 \doteq -c \frac{\partial\phi_{-1}}{\partial\psi} \; .
\end{equation}
%
The first-order flow ${\bf V}_1$ contains both toroidal and poloidal 
components which are self-consistenly calculated by as moments of 
the ensemble averaged first-order distribution, $f_{a1}$ (that is, 
they are computed within the context of neoclassical theory).

\subsection{Equilibrium equation and solution}

The gyrophase average of the zeroth order ensemble-averaged equation 
gives the collisional equilibrium equation:
%
\begin{equation}
\int_0^{2\pi} \frac{d\xi}{2\pi} \fpa_0 = 0 : \qquad 
\left( {\bf V}_0 + \vpp \buv \right) \cdot\nabla f_{a0} = C_a(f_{a0})
\end{equation}
%
where ${\bf v}^\prime = {\bf v} - {\bf V}_0$ is the deviation from
the mean flow velocity.  The exact solution for $f_{a0}$ is a Maxwellian 
in the rotating frame, such that the centrifugal force causes the density 
to vary on the flux surface: 
%
\begin{equation}
f_{a0} = \frac{n_\si(\psi,\theta)}{(2\pi T_\si/m_\si)^{3/2}} 
 \exp\left( -\frac{m_\si (v^\prime)^2}{2 T_\si} \right) 
= n_\si F_{M\si} \; .
\end{equation} 
%
It is important to note here that to account for sonic rotation in the 
equilibrium, we {\bf do not} use the shifted Maxwellian approach 
\cite{waltz:2007b}.  Although the use of a shifted Maxwellian gives 
rise to errors which are probably not significant for typical operating 
parameters, the approach is conceptually incorrect.  Instead, we work 
in the shifted velocity frame ${\bf v}^\prime = {\bf v} - {\bf V}_0$.  
The latter approach was first used in the context of neoclassical 
transport by Hinton and Wong \cite{hinton:1985}. 

Beyond this point, we limit our attention to the {\it moderate 
flow regime}, such that $\rho_* \ll V_0/c_s \ll 1$.  Operationally, 
this means that we will ignore terms quadratic in $V_0/c_s$, whilst 
retaining all terms which are linear in $V_0/c_s$.  Physically, this 
approach will not capture centrifugal effects like the poloidal 
variation of density, since
%
\begin{equation}
n_a(\psi,\theta) \sim n_a(\psi) + {\cal O}(V_0^2/c_s^2) \; ,
\end{equation}
%
but will correctly retain the symmetry-breaking effects of radial
electric field shear, rotation shear drive, and the Coriolis drift.
%
\subsection{The drift-kinetic equation}

Taking a gyroaverage of the first-order ensemble-averaged component 
${\cal A}_1$ gives expressions for the gyroangle-dependent and 
independent distributions, ${\tilde f}_{a1}$ and ${\bar f}_{a1}$:
%
\begin{equation}
\int_0^{2\pi} \frac{d\xi}{2\pi} \fpa_1 = 0 : \qquad 
 f_{a1} = {\tilde f}_{a1} + {\bar f}_{a1} \; , \quad 
{\tilde f}_{a1} = \frac{1}{\Omega_a} \int^\xi d\xi \, \widetilde{{\cal L} f_{a0}}
\end{equation}
%
The function ${\bar f}_{a1}$ is determined by the solution of 
the {\it drift kinetic equation}.  

\subsection{The gyro-kinetic equation}

The gyroaverage of first-order ${\cal F}_1$ gives an expression for 
first-order fluctuating distribution, ${\hat f}_{a1}$, in terms of the 
distribution of the gyrocenters, $H_a({\bf R})$:
%
\begin{equation}
\int_0^{2\pi} \frac{d\xi}{2\pi} \fpf_1 = 0 : \qquad  
{\hat f}_{a1}({\bf x}) = -\frac{e_a {\dphi}({\bf x})}{T_a} f_{a0} + 
H_a(\R) \; ,
\end{equation}
%
where $\x =\R+\brho$ is the particle position, $\brho = \buv\times\vvec^\prime/\wc$ 
is the gyroradius vector, $\wc = e_a B/(m_\si c)$ is the cyclotron frequency, 
and $\R$ (${\bf X}$ in Ref.~\cite{sugama:1998}) is the guiding-center 
position.  The function $H_a(\R)$ ($h_a(\R)$ in Ref.~\cite{sugama:1998}) 
is determined by solution of the nonlinear gyrokinetic equation.  Also 
note that the perturbed potential $\dphi$ appears as ${\hat \phi}$ in 
Ref.~\cite{sugama:1998}.

%---------------------------------------------------------------------
\section{The Gyrokinetic Equation in Detail}

{\bf In what follows, we will drop the prime notation in reference 
to the velocity coordinates}. To bring the gyrokinetic equation into 
a form convenient for numerical integration, we introduce the 
function $h_a(\R)$:
%
\begin{equation}
H_a(\R) = \frac{e_a f_{a0}}{T_\si} \, \Psi_a(\R) + h_a(\R)
\label{eq.bigh}
\end{equation}
%
where $\Psi_a$ (${\hat\psi}_a$ in Ref.~\cite{sugama:1998}) is the 
following gyrophase average (or, more simply, gyroaverage) at 
fixed $\R$:
%
\begin{equation}
\Psi_a(\R) \doteq \left< 
\dphi(\R+\brho)-\frac{1}{c}({\bf V}_0 + \vvec) \cdot 
\delta {\bf A}(\R+\brho) \right>_{\R} \; .
\label{eq.bigpsi}
\end{equation}
%
The gyroaverage can be defined formally as
%
\begin{equation}
\left< z(\R,\xi) \right>_{\R} \doteq \oint \frac{d\xi}{2\pi} \, z(\R,\xi) \; ,
\label{eq.gyroav}
\end{equation}
%
for any function, $z$.    Also, we note 
the following identities 
%
\begin{align}
\buv \cdot\nabla \mathbf{V}_0 = &~\omega_0 \s \\
\buv \cdot\nabla \mathbf{V}_0 + \nabla \mathbf{V}_0 \cdot \buv = &~ 
 \frac{I}{B} \nabla \omega_0 \; ,
\end{align}
%
where $\s$ is the dimensionless vector 
%
\begin{equation}
\s = \frac{1}{\J B} \frac{\partial R}{\partial\theta} \ephi 
- \frac{I}{RB} \nabla R \; .
\end{equation}
%
We use a form of the gyrokinetic equation which can be obtained, 
after some rearrangement, from Eq.~(46) of Ref.~\cite{sugama:1998}.
%
\begin{align}
\frac{\partial h_\si}{\partial t} + \left( v_\parallel \buv + \vd \right)
\cdot\nabla H_\si + \vez \cdot \nabla h_\si 
+ &~\vef \cdot \nabla h_\si \nonumber\\
& + \vef \cdot \left( \nabla f_{a0} + 
 \frac{m_a \vp f_{a0}}{T_a} \frac{I}{B} \nabla \omega_0 \right) 
 = C_a^{GL}\left[ H_a \right] \; .
\label{eq.gkraw}
\end{align}
%
The velocities are  
%
\begin{align}
\vd \doteq &~\frac{\vp^2+\mu B}{\wc B} \, \buv \times\nabla B
+ \frac{2 v_\parallel \omega_0}{\wc} \, \buv \times \s 
 + \frac{4\pi \vp^2}{\wc B^2} \, \buv \times \nabla p \\
\vez \doteq &~\frac{c}{B} \, \buv\times\nabla\phi_{-1}\; , \\
\vef \doteq &~\frac{c}{B} \, \buv\times\nabla\Psi_\si \; .
\end{align}
%
In this result, $\s$ is a dimensionless vector which is 
{\bf not} exactly in the grad-$B$ direction.
The correct form of $s$ cannot be obtained using the shifted 
Maxwellian model.   This form of the drift matches Brizard's
result, and the resulting expression for $\vd \cdot \nabla\psi$ 
matches the familiar result from Hinton and Wong \cite{hinton:1985}.  
By setting $\omega_0=0$, we recover the usual diamagnetic 
rotation ordering.  For normalization purposes it is useful to note that
%
\begin{equation}
\int\dv \, F_{M\si} = 1 \; .
\end{equation}
%
Various terms can be simplified without refering to the geometry 
model.  Within the accuracy of the gyrokinetic ordering, we have
%
\begin{align}
\vez\cdot\nabla h_\si = &~\frac{c}{B} \, \buv\times\nabla\phi_{-1} 
 \cdot\nabla h_\si 
 \sim \omega_0 \frac{\partial h_\si}{\partial\alpha} \; , 
   \label{eq.vedotgradh} \\
\vef\cdot\nabla f_{a0} = &~\frac{c}{B} \, \buv\times\nabla\Psi_a
 \cdot\nabla f_{a0} \sim c\, \frac{\partial f_{a0}}{\partial\psi}
 \frac{\partial\Psi_a}{\partial\alpha} \; ,
   \label{eq.dvdotgradf} \\
\vef\cdot\nabla h_\si = &~\frac{c}{B} \, \buv\times\nabla\Psi_a
 \cdot\nabla h_\si \sim
 c\,\frac{\partial h_\si}{\partial\psi}
 \frac{\partial\Psi_a}{\partial\alpha}-
 c\,\frac{\partial h_\si}{\partial\alpha}
 \frac{\partial\Psi_a}{\partial\psi} \; .
   \label{eq.dvdotgradh}  
\end{align}
%
Using these expression, we find
\begin{align}
\frac{\partial h_\si}{\partial t} 
+ \frac{v_\parallel}{\J B} \frac{\partial H_a}{\partial\theta}
+ \vd \cdot\nabla H_\si + &~\omega_0 \frac{\partial h_a}{\partial\alpha} 
+ c \left[ h_a,\Psi_a \right]_{\psi,\alpha} \nonumber \\
& + c \left( \frac{\partial f_{a0}}{\partial\psi} + \frac{m_a \vp}{T_\si} \frac{I}{B} 
\frac{\partial\omega_0}{\partial\psi} f_{a0} \right) \frac{\partial\Psi_a}{\partial\alpha} 
= C_a^{GL}\left[ H_a \right] \; .
\label{eq.gk}
\end{align}
%
The expansion and simplification of the remaining operators has been 
treated in Chap.~\ref{chap.geometry}.

\subsection{Ordering}

We remark that the gyrokinetic ordering requires that 
%
\begin{equation}
\frac{e_a \Psi_a}{T_a} \sim \frac{h_a}{f_{a0}} 
\sim \frac{\omega - {\bf k}_\perp \cdot {\bf V}_0}{\Omega_{ca}} 
\sim k_\parallel \rho_s \sim \rho_* \; .  
\label{eq.ordrho}
\end{equation}

\subsection{Rotation and rotation shear parameters}

Recalling the definition of the rotation frequency:
%
\begin{equation}
\omega_0 \doteq -c \frac{\partial\phi_{-1}}{\partial\psi} \; , 
\end{equation}
%
we define the {\it Mach number}, the $E_r$ {\it shearing rate} and 
the {\it rotation shearing rate} respectively as
%
\begin{align}
M \doteq &~\frac{\omega_0 R_0}{c_s} 
 \; , \\
\gamma_E \doteq &~-\frac{r}{q} \frac{\partial\omega_0}{\partial r} 
 \; ,\label{eq.ershear}\\
\gamma_p \doteq &~-R_0 \frac{\partial\omega_0}{\partial r} \; .
\end{align}
%
These parameters are defined in this way for legacy reasons and 
are not independent; rather, we have the constraint
%
\begin{equation}
\gamma_p = \frac{q R_0}{r} \gamma_E \; .
\end{equation}

\subsection{Comment on the Hahm-Burrell shearing rate}

\noindent
Note that the shearing rate defined in Eq.~(\ref{eq.ershear}) is not 
in general equal to the familiar {\it Hahm-Burrell shearing rate} 
\cite{burrell:1997} 
%
\begin{equation}
\gamma_{\rm E}^{\rm HB} \doteq \frac{\left(R B_p\right)^2}{B}
\frac{\partial}{\partial \psi} \left( \frac{E_r}{R B_p} \right) \; ,
\end{equation}

\noindent
where $E_r$ is the radial electric field 
%
\begin{equation}
E_r \doteq - \hat{\e}_r \cdot \nabla \phi_{-1} 
    = - |\nabla r| \, \frac{\partial\phi_{-1}}{\partial r} \; . 
\end{equation}

\noindent
In terms of the Miller geometry coefficients, 
$\gamma_{\rm E}^{\rm HB}$ can be written as 
%
\begin{equation}
\gamma_{\rm E}^{\rm HB} = 
  \frac{|\nabla r|}{G_q} \frac{r}{q} \frac{\partial}{\partial r} 
  \left( \frac{c}{\bu} \frac{q}{r} \frac{\partial \phi_{-1}}{\partial r} 
  \right) = 
   \frac{|\nabla r|}{G_q} \, \gamma_{\rm E} \; .
\end{equation}

%---------------------------------------------------------------------
\section{Maxwell equations}

Defining the scalar electromagnetic fields $\dap \doteq \buv \cdot \delta {\bf A}$ 
and $\dbp \doteq \buv \cdot \nabla \times \delta {\bf A}$, we can write an 
equation for each of the fields $(\dphi,\dap,\dbp)$ (see Appendix A, 
\cite{sugama:1998}).  In each case, the species summation runs over all 
species $a$ (ions and electrons).\\
\\
%
{\bf Poisson equation}
%
\begin{equation}
-\nabla_\perp^2 \delta\phi(\x) = 4\pi \sum_\si e z_\si \, \delta n_\si 
  = 4\pi \sum_\si e_a \int \dv \, {\hat f}_{a1}(\x) \; .
\end{equation}
%
{\bf Parallel Amp\`ere's Law}
%
\begin{equation}
-\nabla_\perp^2 \dap(\x) = \frac{4\pi}{c} \sum_\si \delta j_{\parallel,\si} = 
\frac{4\pi}{c} \sum_\si e_a \int\dv \, \vp \, {\hat f}_{a1}(\x) \; .
\end{equation}
%
{\bf Perpendicular Amp\`ere's Law}
%
\begin{equation}
\nabla_\perp \delta B_\parallel(\x) \times \buv 
= \frac{4 \pi}{c} \sum_a \delta {\bf j}_{\perp,a}
= \frac{4 \pi}{c} \sum_a 
e_a \int \dv \, \vvec_\perp {\hat f}_{a1}(\x) 
\end{equation}
%
The righthand sides can be written in terms of $H_a$ according to
%
\begin{align}
\int \dv \, {\hat f}_{a1}(\x) = &~ -\frac{n_a e_a}{T_a} \, \dphi(\x) + 
\int \dv \, H_a(\x-\brho) \; , \\
\int \dv \, \vp \, {\hat f}_{a1}(\x) = &~ \int \dv \, \vp \, H_a(\x-\brho) \\
\int \dv \, \vvec_\perp \, {\hat f}_{a1}(\x) = &~ \int \dv \, 
 \vvec_\perp \, H_a(\x-\brho)
\end{align}

%---------------------------------------------------------------------
\section{Transport Fluxes and Heating}

For each species separately, we define a {\it particle flux}, 
a {\it toroidal angular momentum flux}, an {\it energy flux}, 
and an {\it exchange power density}:
%
\begin{align}
\Gamma_a(r) &~= \fluxa \int\dv \, H_a^*(\R) \, 
  \vef \cdot\nabla r \; , \\
Q_a(r) &~= \fluxa \int\dv \, \, H_a^*(\R) \, 
  \vef\cdot\nabla r \, \frac{1}{2} m_a v^2 \; , \\
\Pi_a(r) &~= \fluxa \int\dv \, H_a^*(\R) \nonumber \\
&~~\left< \left[ m_a R ( {\bf V}_0 + \vvec ) \cdot {\bf e}_\varphi \right] 
  \frac{c}{B} \buv \times \nabla \left[ 
  \dphi(\x)-\frac{1}{c}({\bf V}_0 + \vvec) \cdot 
\delta {\bf A}(\x) \right] \cdot \nabla r \right>_{\R} \\
S_a(r) &~= \fluxa \int\dv \, H_a^*(\R) \, e_a 
\left< \left( \frac{\partial}{\partial t} + {\bf V}_0(\x) 
\cdot \frac{\partial}{\partial\x} \right)  
\left[ \dphi(\x)-\frac{1}{c}({\bf V}_0 + \vvec) \cdot 
\delta {\bf A}(\x) \right] \right>_{\R}
\end{align}

%---------------------------------------------------------------------
\subsection{Ambipolarity and Exchange Symmetries}

By summing the particle fluxes over species and using the Maxwell 
equations, one can prove the exact ambipolarity property
%
\begin{equation}
\sum_a e_a {\bar \Gamma}_a = 0 \; ,
\end{equation}
%
where an overbar denotes a perpendicular spatial and time average 
taken in the flux-tube limit.  Similarly, summing the exchange power 
density over species and using the Maxwell equations, one can prove
the net heating is zero:
%
\begin{equation}
\sum_a {\bar S}_a = 0 \; .
\end{equation}
%
Note that the time-average is only required to prove the exchange 
property, not the ambipolarity property.  Both these conditions are 
in general violated if profile variation is allowed.

%---------------------------------------------------------------------
\section{Entropy production}

The balance equation for entropy production is given by
%
\begin{equation}
\sigma_a 
- \fluxa \int\dv \, \frac{H_a^*}{f_{a0}} \, \frac{\partial H_a}{\partial t}  
+ \fluxa \int\dv \, \frac{H_a^*}{f_{a0}} \, C_a^{GL} 
+ \fluxa \int\dv \, \frac{H_a^*}{f_{a0}} \, D_\tau  
+ \fluxa \int\dv \, \frac{H_a^*}{f_{a0}} \, D_r \rightarrow 0 \; ,
\end{equation}
%
where $D_\tau$ and $D_r$ represent the (artificial) upwind dissipation 
terms added in the numerical discretization.  The function $\sigma_a$ 
is
%
\begin{equation}
\sigma_a \doteq
 \left( \frac{1}{L_{na}} - \frac{3}{2} \frac{1}{L_{Ta}} \right) \Gamma_a
+ \frac{1}{L_{Ta}} \frac{Q_a}{T_a} 
+ \frac{\partial\omega_0}{\partial r} \frac{\Pi_a}{T_a}  
+ S_a \frac{1}{T_a} \; .
\end{equation} 
%
On taking a radial average, and the time-average over sufficiently 
long times, the sum of terms should approach zero.

\section{Simplified fluxes and field equations with operator notation}

\subsection{Operator notation}\label{sec.opnotation}

It is useful at this point to discuss the general spectral 
representation of fields and operators.  First, expanding an 
arbitrary field in a spectral (Fourier) basis gives
%
\begin{equation}
z(\R) \doteq \sum_{\kpv} e^{iS(\R)} \, \tilde{z}(\kpv) \; ,
\end{equation}
%
where $\kpv = \nabla_\perp S$.  In this case, the gyrophase dependence 
in Fourier space is harmonic
%
\begin{equation}
z(\R+\brho) = \sum_{\kpv} e^{iS(\R)} e^{i \kpv \cdot \brho} 
  \, \tilde{z}(\kpv) \; ,
\end{equation}
% 
and the gyroaverage becomes
%
\begin{equation}
\left< z(\R+\brho) \right>_{\R} = \sum_{\kpv} e^{iS(\R)} J_0 (k_\perp\rho_a) \, \tilde{z}(\kpv) \; ,
\end{equation}
%
where $\rho_a \doteq v_\perp/\Omega_{ca}$.  Thus, in real space, the 
gyroaverage can be represented as a linear operator $\gav_{0a}$ whose 
spectral representation is $J_0(k_\perp \rho_a)$.

Now, we can write the field defined in Eq.~\ref{eq.bigpsi} using operator 
notation.  Moreover, we also define a new field which is useful for 
calculation of momentum transport coefficients.  These are
%
\begin{align}
\Psi_a(\R) \doteq &~\gav_{0a} \left[ \dphi(\R) - \frac{\vp}{c} \dap(\R) \right]
 + \frac{v_\perp^2}{\Omega_{ca} c} \gav_{1a} \, \dbp(\R) \; , \\
\mathcal{X}_a(\R) \doteq &~\gav_{2a} \left[ \dphi(\R) - \frac{\vp}{c} \dap(\R) \right]
 + \frac{v_\perp^2}{\Omega_{ca} c} \gav_{3a} \dbp(\R) \; . 
\end{align}
%
Although we will construct explicit discrete 
approximations to the operators $\gav_{0a}$, $\gav_{1a}$ and $\gav_{2a}$
in the next chapter, it can be shown that they have the following 
spectral representations:
%
\begin{align}
\gav_{0a} \rightarrow &~J_0(\gamma_a) \; , \\
\gav_{1a} \rightarrow &~\frac{1}{2} \left[ 
  J_0(\gamma_a) + J_2(\gamma_a) \right] \; , \\
\gav_{2a} \rightarrow &~-i \frac{k_x \rho_a}{2} \left[ 
  J_0(\gamma_a) + J_2(\gamma_a) \right] \; , \\
\gav_{3a} \rightarrow &~i \frac{k_x \rho_a}{\gamma_a^2} \left[ 
  J_0(\gamma_a) - J_1(\gamma_a)/\gamma_a \right] \; ,
\end{align}
%
where $\gamma_a \doteq k_\perp \rho_a$, and $k_x$ is defined explicitly in 
Sec.~\ref{sec.opdisc}.  The expressions above are adapted directly from Sugama \cite{sugama:1998}.  
However, to make use of Sugama's results, we use the following identities, 
where $\boldsymbol{\varphi} = \ephi$:
%
\begin{align}
\kpv\kpv : (R\hat{\boldsymbol{\varphi}}) (\nabla\psi) 
 = &~ \left[ \kpv \cdot  (R\hat{\boldsymbol{\varphi}}) \right] 
      \left[ \kpv \cdot \nabla\psi \right] \; , \\
\kpv \cdot \nabla\psi = &~ -i R B_p \left( |\nabla r| \frac{\partial}{\partial r} 
- \frac{q}{r} \mq \mk \frac{\partial}{\partial\alpha} \right) = R B_p k_x \; , \\
\kpv \cdot  (R\hat{\boldsymbol{\varphi}}) 
 = &~ -i \frac{\partial}{\partial\alpha} \; ,
\end{align}
% 
where Eq.~(\ref{eq.nabla}) has been used to expand $\nabla_\perp$.

\subsection{Maxwell equations: $H_a$-form}

The Maxwell equations are simplest when written in terms of $H_a$:\\
\\
{\bf Poisson equation}
%
\begin{equation}
- \frac{1}{4\pi} \nabla_\perp^2 \dphi = \sum_a e_a 
 \left[ \frac{-n_a e_a}{T_a} \, \dphi + \int \dv \, \gav_{0a} H_a \right] 
\end{equation}
%
{\bf Parallel Amp\`ere's Law}
%
\begin{equation}
- \frac{1}{4\pi} \nabla_\perp^2 \dap = \sum_a e_a \int \dv \, \frac{\vp}{c} 
 \, \gav_{0a} H_a 
\end{equation}
%
{\bf Perpendicular Amp\`ere's Law}
%
\begin{equation}
- \frac{1}{4\pi} \dbp = \sum_a e_a \int \dv \, \frac{v_\perp^2}{\Omega_{ca} c} 
 \, \gav_{1a} H_a  
\end{equation}

\subsection{Maxwell equations: $h_a$-form}

For time-integration purposes, we will need to write the field equations 
in terms of $h_a$:\\
\\
{\bf Poisson equation}
%
\begin{align}
-\frac{1}{4 \pi} \nabla_\perp^2 \dphi 
+ \sum_a n_a \frac{e_a^2}{T_a} & \, \int \dv \, F_{Ma} \, (1-\gav_{0a}^2) \dphi 
\nonumber \\
& - \sum_a n_a \frac{e_a^2}{T_a} \int \dv \, F_{Ma} \gav_{0a} \gav_{1a}
  \frac{v_\perp^2}{\Omega_{ca} c} \dbp 
= \sum_a e_a \int \dv \, \gav_{0a} h_a 
 \label{eq.maxp}
\end{align}
%
{\bf Parallel Amp\`ere's Law}
%
\begin{equation}
-\frac{1}{4 \pi} \nabla_\perp^2 \dap 
+ \sum_a n_a \frac{e_a^2}{T_a} \int \dv \, \frac{\vp^2}{c^2} F_{Ma} \gav_{0a}^2 \dap
= \sum_a e_a \int \dv \, \frac{\vp}{c} \gav_{0a} h_a 
 \label{eq.maxa1}
\end{equation}
%
{\bf Perpendicular Amp\`ere's Law}
%
\begin{align}
\frac{1}{4 \pi} \dbp + \sum_a n_a \frac{e_a^2}{T_a} & \, \int \dv \, F_{Ma} 
\left( \frac{v_\perp^2}{\Omega_{ca} c} \gav_{1a} \right)^2 \dbp
\nonumber \\
&~ + \sum_a n_a \frac{e_a^2}{T_a} \int \dv \, F_{Ma} 
\frac{v_\perp^2}{\Omega_{ca} c} \gav_{1a} \gav_{0a} \dphi
= -\sum_a e_a \int \dv \, \gav_{1a} \frac{v_\perp^2}{\Omega_{ca} c} h_a 
 \label{eq.maxa2}
\end{align}
%
\subsection{Transport coefficients}

Some algebra yields the simplifications:
%
\begin{align}
\Gamma_a(r) &~= \frac{c}{\psi^\prime} \, \fluxa \int\dv \, H_a^*(\R) \, 
  \frac{\partial\Psi_a}{\partial\alpha} \; , 
 \label{eq.flux1}\\
Q_a(r) &~= \frac{c}{\psi^\prime} \, \fluxa \int\dv \, H_a^*(\R) \, 
  \frac{1}{2} m_a v^2 \,
  \frac{\partial\Psi_a}{\partial\alpha} \; , 
 \label{eq.flux2} \\
\Pi_a(r) &~= \frac{c}{\psi^\prime} \, 
   \fluxa \int\dv \, H_a^*(\R) m_a R \left[ 
   \left( V_0 + \vp \frac{B_t}{B} \right) \frac{\partial\Psi_a}{\partial\alpha} 
+ v_\perp \frac{B_p}{B} \frac{\partial\mathcal{X}_a}{\partial\alpha} 
 \right] \; , 
 \label{eq.flux3}  \\
S_a(r) &~= \frac{c}{\psi^\prime} \, 
  \fluxa \int\dv \, H_a^*(\R) \, e_a 
  \left( \frac{\partial}{\partial t} + 
  \omega_0 \frac{\partial}{\partial \alpha} \right) \Psi_a \; .
  \label{eq.flux4}  
\end{align}
