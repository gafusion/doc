\chapter{Maxwell Dispersion Matrix Eigenvalue Solver}

\section{Motivation and Related Work}

In high-beta, strongly shaped plasmas like in the National Spherical Torus
Experiment (NSTX) \cite{ono:2000}, numerous 
branches of closely-spaced unstable eigenmodes exist.  In our 
experience, when modes are closely spaced, it is difficult 
and time-consuming to resolve the most unstable mode using 
the linear initial-value approach.  To overcome these and 
other difficulties encountered in simulating NSTX plasmas,  
we have developed a new, more efficient method to compute 
unstable linear eigenvalues and eigenvectors.  The method is 
valid for tokamak plasmas with arbitrary shape, and can retain 
both the compressional and transverse components of the magnetic 
perturbation.  The new method is a simple extension of the GYRO 
code \cite{candy:2003}, and reuses the existing discretization schemes in both 
real and velocity space.  Existing methods for solving the 
linear gyrokinetic eigenvalue problem fall into two general 
categories: {\it gyrokinetic eigenvalue solvers}, which use 
an iterative approach to compute eigenvalues of the relatively 
large gyrokinetic response matrix \cite{kammerer:2008,bass:2009}, 
and {\it field dispersion-relation solvers}, which compute the 
zeros of the much smaller dielectric matrix using a direct method. 
The former solvers are too expensive for routine linear 
analysis and are not discussed further.  On the other hand, 
there are numerous examples in the literature of the latter 
method, the earliest and still one of the most capable being 
due to Rewoldt \cite{rewoldt:1981}.  Certain solvers distinguish
themselves with a particular capability: for example, global 
capability \cite{falchetto:2003} or the ability to compute 
stable eigenmodes \cite{sugama:1999}.  The present solver is 
unique in that all the linear physics capabilities of GYRO 
can be retained, including pitch-angle collisions (although 
at significantly increased computational expense), global 
effects (since the ballooning transform is not used), finite 
plasma rotation, general plasma shape and compressional magnetic 
perturbations.  The solver is also parallelized, with all costly 
matrix operations (LU decomposition, inverse, matrix-matrix multiply) 
implemented fully in BLAS and LAPACK.  A typical collisionless 
electromagnetic eigenvalue and eigenvector can be computed at 
standard resolution in about 5 seconds on a single 2.66GHz core.  

\section{The Linearized Gyrokinetic Equation}

Just as in the initial-value formulation of the nonlinear gyrokinetic 
equation, here we expand fluctuating quantities in Fourier series, for 
example
%
\begin{equation}
\Psi_a = \sum_n e^{-in\alpha} \Psi_{a,n} \; .
\end{equation}
%
We remark that because of the symmetry properties of the equations, 
it is natural to label toroidal eigenmodes with $k_\theta \rho_s$
rather than $n$, where $k_\theta \doteq n q/r$.  For a single toroidal 
harmonic, the gyrokinetic equation is written symbolically as
%
\begin{equation}
\frac{\partial h_{a,n}}{\partial t} - i(\omega_\theta + \omega_d + \omega_C) H_{a,n} 
-i \omega_* (\frac{e_a f_{0a}}{T_a} \Psi_{a,n}) = 0
\end{equation}
%
where $\omega_\theta$, $\omega_d$ and $\omega_C$ are differential operators:
%
\begin{align}
-i \omega_\theta &~= 
 \vp (\buv \cdot \nabla\theta) \frac{\partial}{\partial\theta} \; , 
\label{eq.omtheta}\\
-i \omega_d &~= -i n (\vd \cdot \nabla\alpha) + 
 (\vd\cdot\nabla r)\frac{\partial}{\partial r} \; , 
\label{eq.omdrift}\\
-i \omega_C &~= - \frac{\nu_a}{2} \frac{\partial}{\partial\xi} 
 \left( 1-\xi^2\right) \frac{\partial}{\partial\xi} \; ,
\label{eq.omcoll}\\
-i \omega_* &~= i \frac{v_{ta}}{a} k_\theta \rho_a \left[ 
\frac{a}{L_{na}} + 
 \left(\frac{v^2}{2 v_{ta}^2}-\frac{3}{2} \right) \frac{a}{L_{Ta}} \right] \; .
\label{eq.omstar}
\end{align}

\section{Construction of the Dispersion Matrix}
The Laplace transform \cite{zayed:1996} of a function $f(t)$, which 
we assume to be differentiable on $(0,\infty)$, is
%
\begin{equation}
\widetilde{f}(s) \doteq {\cal L} f = \int_0^\infty f(t) e^{-st} dt
\end{equation}
%
whenever the integral exists for at least one value of $s$.  In the present 
case, the integral will converge for $s > s_0$, where $s_0$ is the maximum 
linear growth rate. The inversion formula is given by the Bromwich integral
%
\begin{equation}
f(t) \doteq {\cal L}^{-1} \widetilde{f} = \frac{1}{2\pi i} \int_{c-i\infty}^{c+i\infty} 
\widetilde{f}(s) e^{st} ds \; , \quad 0 < t < \infty \; .
\end{equation}
%
It will be convenient to use the variable $\omega=is$ in subsequent formulae.
Now, it is then easy to show that
%
\begin{equation}
\widetilde{H}_a(\omega) = {\cal L} H_{a,n} = 
 \frac{1}{\omega + \omega_\theta + \omega_d +\omega_C} \, ih_a(0) 
+ \frac{\omega-\omega_*}{\omega + \omega_\theta + \omega_d+\omega_C} 
\left( \frac{e_a f_{0a}}{T_a} \widetilde{\Psi}_a \right) \; ,
\end{equation}
%
where $\widetilde{\Psi}_a = {\cal L} \Psi_{a,n}$.
Upon substitution into the Laplace transform of the Maxwell 
equations, we are left with a system of the form
%
\begin{equation}
{\cal M}^{\sigma\sigma^\prime}(\omega) \Phi^{\sigma^\prime}(\omega) = 
 S^\sigma(\omega)
\end{equation}
%
where $\sigma$ and $\sigma^\prime$ are field indices which run from 
1 to 3.  The field matrix and source are given by
%
\begin{align}
{\cal M}^{\sigma\sigma^\prime}(\omega) = &~\delta_{\sigma\sigma^\prime}
L^\sigma - 
 \sum_a \frac{e_a^2}{T_a} \int \dv \, f_{0a} G^{\sigma a} 
\frac{\omega-\omega_*}{\omega+\omega_\theta+\omega_d+\omega_C} 
 G^{\sigma^\prime a} \; , \\
S^\sigma(\omega) = &~\sum_a e_a \int \dv \, G^{\sigma a} 
\frac{1}{\omega+\omega_\theta+\omega_d+\omega_C} 
\, ih_a(0) \; .
\end{align}
%
We have defined the additional 3-index vectors:
%
\begin{align}
\left(\Phi^1,\Phi^2,\Phi^3\right) \doteq &~\left(
\widetilde{\dphi},\widetilde{\dap},\widetilde{\dbp} \right) \; , \\
\left(L^1,L^2,L^3\right) \doteq &~\left(
-\frac{1}{4\pi}\nabla_\perp^2+\sum_a \frac{e_a^2}{T_a} \int\dv \, f_{0a} , \,
\frac{1}{4\pi}\nabla_\perp^2, \,
-\frac{1}{4\pi} 
\right) \; , \\
\left( G^{1a}, G^{2a}, G^{3a} \right) = &~\left(
{\cal G}_{0a},\, 
-\frac{\vp}{c}{\cal G}_{0a}, \,
\frac{v_\perp^2}{\Omega_{ca}c} {\cal G}_{1a} \right) \; .
\end{align}
%
In what follows, we will refer to ${\cal M}(\omega)$ as the 
{\it Maxwell dispersion matrix}.  The roots of the equation 
$\det{\cal M}(\omega) = 0$ correspond to the normal modes of the 
system.  When the velocity integrals are taken along the real velocity 
axes, the integrals used to compute ${\cal M}$ define 
a function of $s$ analytic in region $s > 0$, which corresponds to the 
upper-half $\omega$-plane.  This means that unstable modes can be readily 
computed using the same (real) velocity discretization as in the 
initial-value problem.  Calculation of stable normal modes, on the other 
hand, would require analytic continuation of ${\cal M}(\omega)$ 
into the lower half-plane, $\operatorname{Im}(s) \le 0$, by deformation 
of the contour in velocity space, or perhaps by numerical analytic 
continuation. 

The numerical implementation of the eigenvalue solver, which 
employs the existing GYRO spatial discretization methods, 
is described in detail in the Appendix.  Here, we remark that 
the size of the final matrix problem is small, with 
$\operatorname{rank}({\cal M}) = n_\sigma n_r n_b$, 
with $n_\sigma$ the number of fields, $n_r$ the 
number of radial gridpoints, and $n_b$ the number 
of poloidal finite elements.  For a basic electrostatic case, 
this can be as small as $\operatorname{rank}({\cal M}) = 24$.  
The dominant cost is therefore not computing $\det{\cal M}(\omega)$, 
but rather computing the inverse $P^{-1}$, where $P$ is the matrix 
representation of the propagator $P=\omega+\omega_\theta+\omega_d+\omega_C$.

\section{Discretization and Implementation in LAPACK}

The spatial discretization of the differential operators 
required for construction of the Maxwell field matrix follows 
precisely the same approach as for the GYRO initial-value 
solver, with the various stencils and quadrature methods 
discussed in detail in Ref.~\cite{candy:2003}.  The velocity-space
variables in GYRO are 
%
\begin{equation}
\lambda \doteq \frac{B_{\rm unit}}{v_\perp^2}{B v^2} 
\quad\mbox{and}\quad
\varepsilon \doteq \frac{m_a v^2}{2 T_a}
\end{equation}
%
Fluctuating quantities are evaluated on a species-independent 
mesh with radial nodes $\{r_i\}_{i=1}^{n_r}$, pitch-angle nodes 
$\{\lambda_k\}_{k=1}^{n_\lambda}$, energy nodes 
$\{\varepsilon_\mu\}_{\mu=1}^{n_\varepsilon}$ and orbit-time nodes 
$\{\tau_m\}_{m=1}^{n_\tau}$.  The gyroaverage of the effective 
potential ${\widetilde\Psi}_a$, for species $a$, has the 
discrete representation
% 
\begin{equation}
({\widetilde\Psi}_a)_{ikm}^{\mu} =\sum_{i^\prime\sigma}G^{\sigma a \mu}_{ii^\prime km} 
\Phi^\sigma_{i^\prime km} \; ,
\end{equation}
%
where the three-potential $\Phi^\sigma$ at the same point 
in configuration space is given by the complex Galerkin 
representation
%
\begin{equation}
\Phi_{ikm}^\sigma = \sum_j c_{ij}^\sigma F_{ij}(\theta_{km}) \; .
\end{equation}
%
Here, $c_{ij}^\sigma$ are the so-called {\it blending coefficients},
and $F_{ij}$ the basis functions defined in Sections 5.2 and 
5.3 of Ref.~\cite{candy:2003}.  The propagator has the 
matrix form
%
\begin{equation}
P_{ii^\prime kk^\prime mm^\prime}^{a\mu} = 
\omega \delta_{ii^\prime kk^\prime mm^\prime} 
+ (\omega_\theta)_{ikmm^\prime}^{a\mu} \delta_{ii^\prime kk^\prime} 
+ (\omega_d)_{ii^\prime km}^{a\mu} \delta_{mm^\prime kk^\prime} 
+ (\omega_C)_{ikk^\prime mm^\prime}^{a\mu} \delta_{i i^\prime} \; .
\end{equation}
%
We can write the nonadiabatic distribution ${\widetilde H}_a$ in terms 
of the inverse of the propagator as
%
\begin{equation}
\left(\frac{{\widetilde H}_a}{f_{Ma}}\right)_{ikm}^{a\mu} = 
\frac{e_a n_a}{T_a}
(P^{-1})_{ii^\prime kk^\prime mm^\prime}^{a\mu} 
(\omega-\omega_{*i^\prime k^\prime m^\prime}^{a\mu})\sum_\sigma 
G_{i^\prime i^{\prime\prime} k^\prime m^\prime}^{\sigma a\mu} 
\sum_j c_{i^{\prime\prime} j}^\sigma 
F_{i^{\prime\prime} j} (\theta_{k^\prime m^\prime}) \; .
\end{equation}
%
Constructing the Galerkin projections of all three Maxwell 
equations, using the technique described in Section 5.3 
of Ref.~\cite{candy:2003}, yields the matrix equation
%
\begin{equation}
M_{ii^\prime jj^\prime} ^{\sigma\sigma^\prime} c_{i^\prime j^\prime}^{\sigma^\prime}  
= \left[ A_{ii^\prime jj^\prime}^\sigma \delta_{\sigma\sigma^\prime} -
B_{ii^\prime jj^\prime}^{\sigma\sigma^\prime}(\omega) \right] 
c_{i^\prime j^\prime}^{\sigma^\prime} = 0 \; ,
\end{equation}
%
where
%
\begin{equation}
 A_{ii^\prime jj^\prime}^\sigma = \sum_{km} 
 F_{ijkm}^* L^\sigma_{ii^\prime km} F_{i^\prime j^\prime km}^* 
\end{equation}
%
and
%
\begin{align}
B_{ii^\prime jj^\prime}^{\sigma\sigma^\prime}(\omega) =
&~\sum_{a\mu} \sum_{i^{\prime\prime\prime} k m} \sum_{i^{\prime\prime} k^\prime m^\prime}
\frac{e_a^2 n_a}{T_a} w_{km}^\mu F_{ijkm}^* G_{ii^{\prime\prime\prime} km}^{\sigma a\mu}
(P^{-1})_{i^{\prime\prime\prime} i^{\prime\prime} kk^\prime mm^\prime}^{a\mu}
(\omega-\omega_{*i^{\prime\prime} k^\prime m^\prime}^{a\mu})
G_{i^{\prime\prime} i^{\prime} k^\prime m^\prime}^{\sigma^\prime a\mu}
 F_{i^\prime j^\prime k^\prime m^\prime} \\
= &~\sum_{a\mu} \sum_{p,p^\prime} U_{q p}^{a\mu} (P^{-1})_{p p^\prime}^{a\mu} 
V_{p^\prime q^\prime}^{a\mu} \\
= &~B_{qq^\prime}^{\sigma \sigma^\prime} \; .
\end{align}
%
Here, the weights $w_{km}^\mu$ are the products of the energy, 
pitch-angle and orbit-time weights defined in Eq.~(72) of 
Ref.~\cite{candy:2003}.  In terms of these weights, the flux-surface 
average of the velocity-space integration is written as 
%
\begin{equation}
{\cal F} \int\dv f_{Ma} \Psi_a \rightarrow  
 \sum_{km\mu} w_{km}^\mu (\Psi_a)_{ikm}^\mu 
\quad\mbox{with}\quad
\sum_{km\mu} w_{km}^\mu = 1 \; .
\end{equation}
%
In addition, we have defined the lumped indices 
$p=(i^{\prime\prime\prime},k,m)$, $p^\prime = (i^{\prime\prime},k^\prime
m^\prime)$, $q=(i,j,\sigma)$ and $q^\prime=(i^\prime,j^\prime,\sigma^\prime)$.  
High performance is achieved by computing the inverse $P^{-1}$ using 
the LAPACK routines {\tt ZGETRF} (LU decomposition) followed by 
{\tt ZGETRI} (inverse), with the subsequent matrix triple-product 
$U P^{-1} V$ evaluated using two sequential calls to the 
BLAS {\tt ZGEMM} kernel.  Finally, $\det (M)$ is computed by 
the formula 
%
\begin{equation}
\det (M) = \pm \prod_q L_{qq}
\end{equation}
%
where $L$ is the lower-triangular matrix returned by 
the {\tt ZGETRF} factorization.  The upper (lower) sign 
is taken if an even (odd) number of row permutations 
were made in the factorization.


